\documentclass[conference]{IEEEtran}



% correct bad hyphenation here
% \hyphenation{op-tical net-works semi-conduc-tor}


\begin{document}

\title{CPEN 442 --- Assignment 3}


% author names and affiliations
% use a multiple column layout for up to three different
% affiliations
% % % % CHANGE: Add your aliases here
\author{\IEEEauthorblockN{Matthew Kuo, Anson Wong}
\IEEEauthorblockA{Department of Computer Science\\
University of British Columbia\\
}
\and
\IEEEauthorblockN{Matt Labbe, Kevin Lim}
\IEEEauthorblockA{Electrical and\\Computer Engineering\\
University of British Columbia\\
}
}


% make the title area
\maketitle

% As a general rule, do not put math, special symbols or citations
% in the abstract
\begin{abstract}
Four CPEN442 students test their knowledge of network programming and exercise understanding of the principles of public key cryptography. 
\end{abstract}



\section{Introduction}
% no \IEEEPARstart
% We built the program with simplicity in mind. Thus, we chose to do the assignment in Python using a command line interface for the server and client. The 

\section{Data Transmission}
We use a TCP/IP connection both to send and receive data. The protocol requires that there exists a server operating before a client can connect. The connection will only be established given the correct port number and IP address are supplied. We transmit data in bytes. 
\section{Mutual Authentication Protocol}
\section{Key Derivation from Shared Secret}
\section{Production Parameters}
\section{Code Particulars}




\section{Conclusion}



% that's all folks
\end{document}


